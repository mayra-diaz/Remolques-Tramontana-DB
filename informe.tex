\documentclass[12pt]{report}
\usepackage[spanish]{babel}
\usepackage[utf8x]{inputenc}
\usepackage{fancyhdr}   
\usepackage{amsmath}
\usepackage{mathtools}
\usepackage{hyperref}
\usepackage[shortlabels]{enumitem}
\usepackage{setspace}
\usepackage{graphicx}
\usepackage{wrapfig}
\usepackage{lscape}
\usepackage{rotating}
\usepackage{epstopdf}
\usepackage[hidelinks]{hyperref}
\usepackage{listings}
\usepackage[dvipsnames]{xcolor}
\usepackage{enumitem}
\usepackage{multicol}
\usepackage{makeidx}
\usepackage{pdflscape}
\usepackage{afterpage}
\usepackage{float}
\usepackage[toc,page]{appendix}
\usepackage[htt]{hyphenat}

\usepackage{tikz}
\usepackage{pgfplots}

\usepackage{titlesec}
\titleformat{\chapter}
   {\huge\bfseries\scshape}{\thechapter.}{1em}{}

\graphicspath{ {media/} }

\pgfplotsset{compat=newest}
\usepgfplotslibrary{units}

\setcounter{tocdepth}{2}
\setcounter{secnumdepth}{3}

\definecolor{codegreen}{rgb}{0,0.6,0}
\definecolor{codegray}{rgb}{0.5,0.5,0.5}
\definecolor{codepurple}{rgb}{0.58,0,0.82}
\definecolor{backcolour}{rgb}{0.98,0.98,0.98}

\lstdefinestyle{NiceStyle}{
  backgroundcolor=\color{backcolour},   
  commentstyle=\color{codepurple},
  keywordstyle=\color{blue},
  numberstyle=\tiny\color{codegray},
  stringstyle=\color{codegreen},
  basicstyle=\ttfamily\footnotesize,
  breakatwhitespace=false,         
  breaklines=true,                 
  captionpos=b,                    
  keepspaces=true,                 
  numbers=left,                    
  numbersep=5pt,                  
  showspaces=false,                
  showstringspaces=false,
  showtabs=false,                  
  tabsize=2
}

\lstdefinestyle{TableStyle}{
  backgroundcolor=\color{backcolour},   
  keywordstyle=\color{blue},
  basicstyle=\ttfamily\footnotesize,
  frame=single,
  breakatwhitespace=true,         
  breaklines=true,                 
  keepspaces=true,                                
  showspaces=false,                
  showstringspaces=false,
  showtabs=false,                  
  tabsize=2
}

\author{Mayra Díaz Tramontana 201910147 \and 
	            Paolo André Morey Tutiven 201910236 \and
	            Joaquín Elías Ramírez Gutiérrez 201910277}
\title{Proyecto - Base de Datos 1}


% encabezados
\lhead[]{CS2701- UTEC \\Base de Datos I}
\chead[]{2020-2}
\rhead[]{Prof. Teófilo Chambilla \\tchambilla@utec.edu.pe}
\renewcommand{\headrulewidth}{0.5pt}
\setlength{\headheight}{28pt} 

% pie de pagina
\rfoot[]{}
\renewcommand{\footrulewidth}{0pt}
\renewcommand{\labelenumii}{\theenumii}
\renewcommand{\theenumii}{\theenumi.\arabic{enumii}.}

% espacio entre párrafos
\setlength{\parskip}{1em}

\pagestyle{fancy}


\begin{document}
\onehalfspacing

\begin{titlepage}
  \centering
  {\bfseries\Large Universidad de Ingeniería y Tecnología\par}
  \vspace{0.5cm}
  {\scshape\Large Ciencia de la Computación\par
   \scshape\large Base de Datos I \par}
  \vspace{0.5cm}
  {\scshape\huge Base de Datos para la empresa REMOLQUES TRAMONTANA S.A.C\par}
  \vspace{2mm}
  \begin{figure}[h!]
    \centering
    \includegraphics[width=220px]{RT_logo.jpeg}
  \end{figure}
  \vspace{0.5cm}
  \vfill
  {\large Autores: \par}
  { Mayra Díaz Tramontana 201910147\par}
  { Paolo André Morey Tutiven 201910236\par}
  { Joaquín Elías Ramírez Gutiérrez 201910277\par}
  \vfill
  {\large Diciembre 2020 \par}
\end{titlepage}
\thispagestyle{fancy}

\tableofcontents
\newpage


\chapter{Requisitos}
\section{Introducción}
Remolques Tramontana es una empresa con más de 20 años de experiencia en el rubro de la Fabricación, Reparación y Alquiler de Remolques y Semiremolques para carga en general, especialmente para cargas especiales, sobredimensionadas, peligrosas y de líquidos. Actualmente, su enfoque está en la fabricación de semiremolques de carga pesada, estos son a pedido y personalizados según la necesidad del cliente.

\begin{itemize}
  \item \textbf{Razón Social}: REMOLQUES TRAMONTANA S.A.C.
  \item \textbf{RUC}: 20514038199
  \item \textbf{Dirección}: Av. Ramiro Priale Mz A, Lt 10B, Huachipa, Lurigancho, Lima
  \item \textbf{Página Web}: \href{https://remolquestramontana.com/}{https://remolquestramontana.com/}
  \item \textbf{Facebook}: \href{https://www.facebook.com/RemolquesTramontana}{https://www.facebook.com/RemolquesTramontana}
\end{itemize}

\section{Descripción general de la organización/problema.}
Remolques Tramontana actualmente tiene diversos problemas en la organización de la data de la empresa. La empresa tiene todas sus transacciones e información en hojas de cálculo en Excel, este método hace que las consultas sobre la empresa sean engorrosas y díficiles de realizar. Al ser una fábrica de maquinaria grande, la cantidad de productos, servicios y herramientas compradas por proyecto resulta alta. Además, puesto que llevan más de dos décadas en el mercado, su cartera de clientes es amplia y por ende, su historial también. Por lo mencionado, hacer un seguimiento a cada proyecto, compra, empresa, proveedor y demás, resulta un trabajo extra para la empresa, el cual podría ser optimizado al utilizar una base de datos.

Otro problema de la empresa es la pérdida de información e inseguridad sobre la misma. Puesto que no tienen las hojas de cálculo actualizadas en la nube, todos los cambios realizados y el ingreso de data se maneja localmente por las personas encargadas. Estas además, tienen que compartirse y enviarse distintas versiones. Este proceso ha ocasionado en múltiples ocasiones pérdida de datos importantes. Por otro lado, al tener que pasar estos archivos (que son pesados), cada vez que otra persona en la empresa quiera verlos o manipularlos, hace perder mucho tiempo; pues no solo demoran en cargar, sino que no siempre los empleados cuentan con un internet estable que les permita realizar dicho procedimiento.

\section{Necesidad/usos de la base de datos.}
Remolques Tramontana tiene varios problemas que derivan del método actual del manejo de información. No cuentan con una base de datos ni data respaldada en la nube. La empresa afirma que encontrar información específica dentro de los documentos resulta pesado y toma mucho tiempo. Además, que sienten que su información no está segura. Usar una base de datos optimizaría sus procesos y transacciones. Esto reduciría el trabajo de las personas encargadas de la administración, y por ende aumentaría la productividad.

\section{¿Cómo resuelve el problema hoy?}
Actualmente resuelven el problema teniendo varias versiones de las hojas de cálculo en distintas computadoras e intercambiando versiones actualizadas entre la parte administrativa.

\subsection{¿Cómo se almacenan/procesan los datos hoy?}
Todos los datos son almacenados y procesados en hojas de Excel.

\subsection{Flujo de datos}
El ingreso de los datos se hace únicamente de forma manual por distintos empleados de la empresa. Al tener varias tablas, la información debe ser completada en todas independientemente, pues no están conectadas. Esto ocasiona que se ingrese de distinta forma y después no se reconozcan adecuadamente estos datos.

\section{Descripción detallada del sistema}

\subsection{Objetos de información actuales}
La información almacenada en las hojas de cálculo actuales será insertada en la base de datos que se realizará en este proyecto.

\subsection{Características y funcionalidades esperadas}
La empresa espera que el sistema sea más seguro, rápido y fácil de utilizar. Se espera que la manipulación de la data sea accedida únicamente por el área de administración y que se realicen distintas consultas en la data, principalmente en base a la aplicación de filtros.

\subsection{Tipos de usuarios existentes/necesarios}
Remolques Tramontana cuenta con 5 áreas:
\begin{enumerate}
    \item Gerencia
    \item Comercio
    \item Logística
    \item Fabricación
    \item Seguridad y limpieza
\end{enumerate}

El área de Gerencia debe tener acceso a la manipulación y visualización total de la data. Y las áreas Comercio y Logística debe tener acceso parcial a la misma.

          
\subsection{Tipos de consulta, actualizaciones}
Las consultas más frecuentes solicitadas por la empresa son:
\begin{itemize}
\subsection{Descripción del tipo de consultas seleccionadas}
 \item ¿Cuántas horas en total trabajaron los empleados con los dos sueldos más altos del área de fabricación?
\item ¿Cuáles son los bienes de determinada categoría, que cuentan con un stock mayor a 0 y han sido comprados en el año \textit{x} por un precio unitario mayor a \textit{y} soles?
\item ¿Qué máquinas han utilizado bienes medidos en unidades \textit{x} por una cantidad total mayor a \textit{y} en una determinada fecha?
 
\end{itemize}

Las actualizaciones más frecuentes serán:
\begin{itemize}
    \item Compras realizadas.
    \item Trabajo de los empleados del área de fabricación.
\end{itemize}

\subsection{Tamaño estimado de la base de datos}
\emph{Calculado en base a bytes}
\begin{table}[H]
  \tiny
  \begin{tabular}{cc|c|c|c|}
  \hline
  \multicolumn{1}{|c|}{\textbf{Tabla}}    & \textbf{Longitud Atributos}               & \textbf{Longitud Registro} & \textbf{Cantidad} & \textbf{Total Registro} \\ \hline
  \multicolumn{1}{|c|}{ContactoCliente}   & 4+80+4+60+8                               & 156                        & 30                & \textbf{4680}           \\ \hline
  \multicolumn{1}{|c|}{ContactoProveedor} & 4+80+4+60+8                               & 156                        & 200               & \textbf{31200}          \\ \hline
  \multicolumn{1}{|c|}{Cliente}           & 8+100+80                                  & 188                        & 30                & \textbf{5640}           \\ \hline
  \multicolumn{1}{|c|}{Proveedor}         & 8+100+80+150+1                            & 339                        & 200               & \textbf{67800}          \\ \hline
  \multicolumn{1}{|c|}{Empleado}          & 4+80+4+60+150+4+8+25+25+8                 & 368                        & 15                & \textbf{5520}           \\ \hline
  \multicolumn{1}{|c|}{Dirige}            & 4+8+1                                     & 13                         & 0                 & \textbf{0}              \\ \hline
  \multicolumn{1}{|c|}{Compra}            & 4+100+8+8+2+4                             & 126                        & 0                 & \textbf{0}              \\ \hline
  \multicolumn{1}{|c|}{Trabajo}           & 8+1+150+4+4                               & 167                        & 0                 & \textbf{0}              \\ \hline
  \multicolumn{1}{|c|}{Maquina}           & 4+150                                     & 154                        & 15                & \textbf{2310}           \\ \hline
  \multicolumn{1}{|c|}{Requiere}          & 100+8+1+2                                 & 111                        & 0                 & \textbf{0}              \\ \hline
  \multicolumn{1}{|c|}{Carroceria}        & 4+20+8+2+25+8+8+8+2+2+15+8+8+8+4+20+1+8+4 & 163                        & 40                & \textbf{6520}           \\ \hline
  \multicolumn{1}{|c|}{Asociado}          & 4+4+100                                   & 108                        & 0                 & \textbf{0}              \\ \hline
  \multicolumn{1}{|c|}{Bien}              & 100+1+15+2+20                             & 138                        & 250               & \textbf{34500}          \\ \hline
  \multicolumn{1}{l}{}                    & \multicolumn{1}{l|}{}                     & \multicolumn{2}{c|}{\textbf{Total}}            & 158170                  \\ \cline{3-5} 
  \end{tabular}
  \end{table}
Se estima un tamaño inicial de 158.170 KB (kilobytes).


\begin{table}[]
  \tiny
  \begin{tabular}{cc|c|c|c|}
  \hline
  \multicolumn{1}{|c|}{\textbf{Tabla}}    & \textbf{Longitud Atributos}               & \textbf{Longitud Registro} & \textbf{Cantidad} & \textbf{Total Registro} \\ \hline
  \multicolumn{1}{|c|}{ContactoCliente}   & 4+80+4+60+8                               & 156                        & 3                 & \textbf{468}            \\ \hline
  \multicolumn{1}{|c|}{ContactoProveedor} & 4+80+4+60+8                               & 156                        & 2                 & \textbf{312}            \\ \hline
  \multicolumn{1}{|c|}{Cliente}           & 8+100+80                                  & 188                        & 3                 & \textbf{564}            \\ \hline
  \multicolumn{1}{|c|}{Proveedor}         & 8+100+80+150+1                            & 339                        & 2                 & \textbf{678}            \\ \hline
  \multicolumn{1}{|c|}{Empleado}          & 4+80+4+60+150+4+8+25+25+8                 & 368                        & 1                 & \textbf{368}            \\ \hline
  \multicolumn{1}{|c|}{Dirige}            & 4+8+1                                     & 13                         & 2550              & \textbf{33150}          \\ \hline
  \multicolumn{1}{|c|}{Compra}            & 4+100+4+8+2+8                             & 126                        & 240               & \textbf{30240}          \\ \hline
  \multicolumn{1}{|c|}{Trabajo}           & 8+1+150+4+4                               & 167                        & 1700              & \textbf{283900}         \\ \hline
  \multicolumn{1}{|c|}{Maquina}           & 4+150                                     & 154                        & 1                 & \textbf{154}            \\ \hline
  \multicolumn{1}{|c|}{Requiere}          & 100+8+1+2                                 & 111                        & 3400              & \textbf{377400}         \\ \hline
  \multicolumn{1}{|c|}{Carroceria}        & 4+20+8+2+25+8+8+8+2+2+15+8+8+8+4+20+1+8+4 & 163                        & 12                & \textbf{1956}           \\ \hline
  \multicolumn{1}{|c|}{Asociado}          & 4+4+100                                   & 108                        & 30                & \textbf{3240}           \\ \hline
  \multicolumn{1}{|c|}{Bien}              & 100+1+15+2+20                             & 138                        & 6                 & \textbf{828}            \\ \hline
  \multicolumn{1}{l}{}                    & \multicolumn{1}{l|}{}                     & \multicolumn{2}{c|}{\textbf{Total}}            & 733258                  \\ \cline{3-5} 
  \end{tabular}
  \end{table}
Se estima un crecimiento semestral de 733.258 KB (kilobytes).


\section{Objetivos del proyecto}
Los objetivos principales del proyecto son:
\begin{enumerate}
    \item Diseño adecuado de una base de datos para la empresa Remolques Tramontana.
    \item Implementación de una base de datos que permita la escalabilidad de la empresa.
    \item Organizar la data de la empresa de tal manera que facilite su análisis y actualización.
    \item Optimizar la base de datos implementada y reducir los tiempos de respuesta y escritura.
\end{enumerate}


\section{Referencias del proyecto}
\begin{itemize}
  \item  Weddel, G. (2011). \textit{Translating Entity-Relationship to Relational Tables.} Personal Collection of G.
  Weddel, David R. Cheriton School of Computer Science University of Waterloo. 
  Recuperado de: \\  \href{https://bit.ly/2WsTmqD}{https://bit.ly/2WsTmqD}
  \item Ramakrishnan, R. \& Gehrke, J. (2003). \textit{Database management systems: (3rd ed)}. Boston: McGraw-Hill.
\end{itemize}

\section{Eventualidades}
\subsection{Problemas que pudieran encontrarse en el proyecto}
\subsubsection{Data}
Puesto que la información en los cuadros de Excel muchas veces está incompleta por las múltiples pérdidas de data, y que la organización y relación de estas no está bien definida, rellenar la base de datos con datos previos podría resultar engorroso. Además, no se podrá tener un registro histórico de la información de la empresa.

\subsubsection{Carrorecía}
Puesto que la placa se saca después de tener la carrocería terminada, podría ser un problema al momento de registrar las compras asociadas a una determinada carrocería. Esto es porque las compras se hacen al momento de la fabricación, por eso probablemente se deba implementar un id. De esta manera se evitaría una posible pérdida de data.

\subsection{Límites y alcances del proyecto}
\textbf{Alcances}:\\
El proyecto presentará una forma de organización de data óptima, que representa una solución para la empresa Remolques Tramontana.

\textbf{Limitaciones}:\\
La data de la empresa actualmente está desorganizada e incompleta.



\chapter{Entidad-Relación}
Para el diagrama del  relacional se hace uso del software draw.io.

\section{Reglas semánticas}
\begin{itemize}
  \item Los clientes y proveedores son únicamente personas jurídicas.
  \item Por cada cliente y proveedor existe mínimo una persona dentro de dicha empresa con la que se realizan las transacciones y toda la comunicación en general.
  \item Las personas de contacto con las empresas pertenecen a una sola empresa. Se necesita saber su dni, celular, correo electrónico y nombre.
  \item Se venden carrocerías, cada venta es de una sola carrocería.
  \item Las carrocerías tienen una placa y código vin único. También tienen una carga útil, peso neto, peso bruto (suma de los dos primeros), color, categoría, tipo de carrocería, número de ejes y de llantas, ancho, largo y alto.
  \item Todas las ventas/compras se realizan por facturas, las que se identifican por su número.
  \item Las carrocerías son hechas a pedido, son personalizadas y se vende una por factura.
  \item No hay stock de carrocerías, pero sí de algunos materiales.
  \item Existen materiales y servicios que son comprados específicamente para una determinada carrocería, es necesario identificar estas compras.
  \item Para fabricar una carrocería se requieren productos y servicios. Estos también están agrupados por categorías.
  \item Los proveedores son únicamente o de productos o de servicios.
  \item Siempre se almacena la ubicación, razón social y RUC de las empresas.
  \item Los empleados están dividos en áreas. De cada uno se necesita saber su fecha de nacimiento, número de cuenta BCP, cargo, sueldo, dirección, correo electrónico, nombre, celular y dni.
  \item Se necesita contabilizar las horas de trabajo de los trabajadores en fabricación, estos utilizan la maquinaria.
  \item Las máquinas funcionan independientemente, se utiliza una por trabajo.
  \item En un día, un empleado puede trabajar con varias máquinas haciendo distintos trabajos.
  \item La mayoría de máquinas utilizan materiales. Las maquinarias se identifican por un código y tienen una descripción.
\end{itemize}

\section{Entidad-Relación}
\begin{figure}[H]
    \centering
    \includegraphics[width=\textwidth]{Entidad-Relacion-RT.pdf}
\end{figure}

El diagrama fue realizado con el software draw.io. Para apreciarlo mejor, consultar el \hyperlink{A:MER}{ \textbf{Anexo A}}

\section{Especificaciones y consideraciones sobre el modelo}
\textbf{\large Persona}
\begin{enumerate}[label=\arabic*)]
  \item \textbf{Especificaciones:} Almacena la información de una persona vinculada a la empresa. Cada persona se identifica con su dni.
  \item \textbf{Consideraciones:} Es superclase de Contacto y Empleado. No hay solapamiento entre dichas entidades, hay cobertura.
\end{enumerate}
\textbf{\large Empleado}
\begin{enumerate}[label=\arabic*)]
  \item \textbf{Especificaciones:} Almacena la información de un empleado dentro de la empresa, quienes están agrupados en áreas.
  \item \textbf{Consideraciones:} Es subclase de Persona.
\end{enumerate}
\textbf{\large Contacto}
\begin{enumerate}[label=\arabic*)]
  \item \textbf{Especificaciones:} Almacena la información de un contacto proveniente de un cliente o un proveedor.
  \item \textbf{Consideraciones:} Es subclase de Persona, solo puede estar asociado a una empresa.
\end{enumerate}
\textbf{\large Empresa}
\begin{enumerate}[label=\arabic*)]
  \item \textbf{Especificaciones:} Almacena la información de la empresa con la que se relaciona Remolques Tramontana, se identifican con el RUC.
  \item \textbf{Consideraciones:} Es superclase de Cliente y Proveedor. No hay solapamiento entre dichas entidades, hay cobertura.
\end{enumerate}
\textbf{\large Cliente}
\begin{enumerate}[label=\arabic*)]
  \item \textbf{Especificaciones:} Almacena la información del cliente que solicita una carrocería.
  \item \textbf{Consideraciones:} Es subclase de Empresa. 
\end{enumerate}
\textbf{\large Proveedor}
\begin{enumerate}[label=\arabic*)]
  \item \textbf{Especificaciones:} Almacena la información del proveedor que le brinda productos o servicios a Remolques Tramontana. Un proveedor puede ser o de productos o de servicios.
  \item \textbf{Consideraciones:}  Es subclase de Empresa.
\end{enumerate}
\textbf{\large Carrocería}
\begin{enumerate}[label=\arabic*)]
  \item \textbf{Especificaciones:} Almacena la información del producto solicitado por un cliente.
  \item \textbf{Consideraciones:} Cada carrocería es única y tiene características distintas, pues son hechas a pedido y personalizadas. Se identifican por la placa y el código vin.
\end{enumerate}
\textbf{\large Compra}
\begin{enumerate}[label=\arabic*)]
  \item \textbf{Especificaciones:} Almacena la información de las compras realizadas por producto y proovedor.
  \item \textbf{Consideraciones:} Cada compra es realizada a un único proveedor. Es débil de Bien porque cada compra identifica a un bien dentro de una factura, por lo que la llave es el bien que se está adquiriendo (producto o servicio), y el número de factura de la compra.
\end{enumerate}
\textbf{\large Bien}
\begin{enumerate}[label=\arabic*)]
  \item \textbf{Especificaciones:} Almacena productos y servicios adquiridos por la empresa.
  \item \textbf{Consideraciones:} Están clasificados, algunos pueden tener stock. Puede ser o producto o servicio, y están medidos en distintas unidades (kilos, galones, entre otros).
\end{enumerate}
\textbf{\large Trabajo}
\begin{enumerate}[label=\arabic*)]
  \item \textbf{Especificaciones:} Almacena la información de los trabajos realizados por los empleados de fabricación por día.
  \item \textbf{Consideraciones:} Un empleado puede trabajar con más de una máquina al día. Algunos trabajos requieren el uso de ciertos bienes. Es necesario contabilizar las horas de uso de las máquinas y de trabajo de los empleados.
\end{enumerate}
\textbf{\large Máquina}
\begin{enumerate}[label=\arabic*)]
  \item \textbf{Especificaciones:} Almacena la información de las máquinas poseídas por la empresa.
  \item \textbf{Consideraciones:} Se identifican por un código único.
\end{enumerate}


\chapter{Modelo Relacional}
 \section{Modelo Relacional}
\begin{figure}[H]
    \centering
    \includegraphics[width=\textwidth]{Modelo-Relacional-RT.pdf}
\end{figure}

El diagrama fue realizado con el software draw.io. Para apreciarlo mejor, consultar el \hyperlink{A:MR}{ \textbf{Anexo B}}

\begin{itemize}
    \item \textbf{\textcolor{ForestGreen}{ContactoCliente}}(
    \textcolor{Blue}{\underline{dni:int},  \textcolor{ForestGreen}{Cliente}.RUC,  nombre:string, celular:int,\\ correo:string})
    
    \item \textbf{\textcolor{ForestGreen}{ContactoProveedor}}(
    \textcolor{Blue}{\underline{dni:int},  \textcolor{ForestGreen}{Proveedor}.RUC,  nombre:string, celular:int, correo:string})
    
    \item \textbf{\textcolor{ForestGreen}{Empleado}}(
    \textcolor{Blue}{\underline{dni:int}, nombre:string,  celular:int, correo:string,  direccion:string,  nacimiento:date, nCuenta:int, area:string, cargo:string, sueldo:float})
    
    \item \textbf{\textcolor{ForestGreen}{Cliente}}(
    \textcolor{Blue}{\underline{RUC:int}, ubicacion:string, razonSocial:string})
    
    \item \textbf{\textcolor{ForestGreen}{Proveedor}}(
    \textcolor{Blue}{\underline{RUC:int}, ubicacion:string, razonSocial:string, descripcion:string, esDeProducto:boolean})
    
    \item \textbf{\textcolor{ForestGreen}{Compra}}(
    \textcolor{Blue}{\underline{nFactura:int}, \underline{\textcolor{ForestGreen}{Bien}.descripcion:string}, fecha:date, precioUnitario:float, cantidad:int, \textcolor{ForestGreen}{Proveedor}.RUC})
    
    \item \textbf{\textcolor{ForestGreen}{Bien}}(
    \textcolor{Blue}{\underline{descripcion:string}, esProducto:bool, clasificacion:string, stock:int, unidades:string})
    
    \item \textbf{\textcolor{ForestGreen}{Carroceria}}(
    \textcolor{Blue}{\underline{id:int}, placa:string vin:string, categoria:string, carroceria:string, largo:float, ancho:float, alto:float, nEjes:int, nLlantas:int, color:string, cargaUtil:float, pesoNeto:float, \textcolor{ForestGreen}{Cliente}.RUC:int, nFactura:int, precio:float, medioPago:string, alContado:bool, deuda:float, fecha:date})
    
    \item \textbf{\textcolor{ForestGreen}{Trabajo}}(
    \textcolor{Blue}{\underline{fecha:datetime}, \underline{numero:int}, descripcion:string, horas:float, \textcolor{ForestGreen}{Maquina}.codigo.int})
    
    \item \textbf{\textcolor{ForestGreen}{Maquina}}(
    \textcolor{Blue}{\underline{codigo:int}, descripcion:string})
    
    \item \textbf{\textcolor{ForestGreen}{Dirige}}(
    \textcolor{Blue}{\underline{\textcolor{ForestGreen}{Empleado}.dni:int}, \underline{\textcolor{ForestGreen}{Trabajo}.fecha:datetime}, \underline{\textcolor{ForestGreen}{Trabajo}.numero:int}})
    
    \item \textbf{\textcolor{ForestGreen}{Requiere}}(
    \textcolor{Blue}{\underline{\textcolor{ForestGreen}{Trabajo}.fecha:datetime}, \underline{\textcolor{ForestGreen}{Trabajo}.numero:int},
    \underline{\textcolor{ForestGreen}{Bien}.descripcion:string}, cantidad:int})
    
    \item \textbf{\textcolor{ForestGreen}{Asociado}}(
    \textcolor{Blue}{\underline{\textcolor{ForestGreen}{Carroceria}.placa:string}, \underline{\textcolor{ForestGreen}{Compra}.nFactura:int}, \\ \underline{\textcolor{ForestGreen}{Compra.Bien}.descripcion:string}})
\end{itemize}


\section{Especificaciones de transformación}
\subsection{Entidades}
\textbf{Carrocería} \\
Es una entidad que se identifica con un id serial (llave primaria), esto es porque al momento de empezar la fabricación de una carrocería aún no se tiene la placa ni el vin. Tiene también una llave foránea que es el RUC de un cliente. Con esto obviamos la relación que existía entre Carrorecería y Cliente, que es Adquiere. Además, cuenta con los siguientes atributos: placa, vin, categoría, carrocería, peso neto, carga útil, color, largo, alto, ancho, número de ejes y número de llantas. Por último, cuenta con peso bruto, que es un atributo derivado. \\\\
\textbf{Bien} \\
Es una entidad que tiene como llave primaria una descripción. Además, sus demás atributos son los sigueintes: unidades, stock, clasificación y esProducto.\\\\
\textbf{Trabajo} \\ 
Es una entidad que tiene como llave primaria un número y una fecha. Cuenta también con un número de horas y una descripción. Además, cuenta con una llave foránea que es el código de Máquina, por lo que obviamos en el modelo relacional la relación que existía entre ellas, que es Utiliza. \\\\
\textbf{Máquina} \\
Es una entidad que se identifica mediante un código (llave de primaria) y además cuenta con un atributo llamado descripción.
\subsection{Entidades débiles}
\textbf{Compra} \\
Es una entidad débil de la entidad Bien, por lo que su llave depende de la llave primaria de bien (descripción, obviamos la relación vieneDe del modelo ER) y su propia llave, que es número de factura. Existe también una llave foránea que es el RUC del proveedor y de esta forma, ignoramos la relación provee en nuestro modelo relacional. Además, cuenta con los siguientes atributos adicionales: fecha, precio unitario, precio total y cantidad.

\subsection{Entidades superclase/subclases}
\textbf{Persona (superclase)}\\
Una persona es identificada mediante un único DNI. Además, cuenta con los atributos de nombre, correo electrónico y número de celular. Debido a que en nuestro modelo no hay solapamiento de los hijos de esta entidad y existe cobertura, decidimos obviar esta entidad en el modelo Entidad-Relación e incluimos sus atributos en Empleado y Contacto. \\\\
\textbf{Empleado (subclase)}\\
Su llave primaria depende de la Superclase Persona, siendo esta el DNI y cuenta con los siguientes atributos adicionales: nacimiento, dirección, número de cuenta, sueldo, cargo en la empresa y el área en el que trabaja. \\\\
\textbf{Contacto (subclase)}\\
Es la segunda clase dependiente de Persona teniendo como llave primaria DNI, al igual que Empleado. Además, tiene una llave foránea que es el RUC de empresa, debido a esto, obviamos la relación comunica del modelo entidad relación al modelo relacional. No cuenta con atributos adicional más que los de la Superclase Persona. Puesto que la entidad Empresa también fue obviada, se ha dividido Contacto en dos tablas:
\begin{itemize}
    \item ContactoCliente
    \item ContactoProveedor
\end{itemize}
Estas tablas almacenan el RUC ya sea de cliente o de proveedor, mantienen el resto de atributos igual.
\textbf{Empresa (superclase)}\\
Las empresas se identifica mediante un RUC único para cada una de ellas. Además, es una superclase y cuenta con otros dos atributos que son ubicación y razón social. No hay solapamiento entre las subclases y existe cobertura.Sin embargo, existe una relación con Contacto, este problema se resuelve diviendo dicha entidad. Por lo tanto, se obvia la existencia de Empresa.\\\\
\textbf{Cliente (subclase)}\\
Es una subclase de Empresa, mas en el modelo relacional se obvia esta característica por la eliminación de la entidad Empresa. Su llave primaria es el RUC de empresa y cuenta con los atributos ubicación y razón social.\\\\
\textbf{Proveedor (subclase)} \\
Es una subclase de Empresa, mas en el modelo relacional se obvia esta característica por la eliminación de la entidad Empresa. Su llave primaria es el RUC de empresa y cuenta con los atributos ubicación, razón social, esDeProducto y descripción. \\\\
\subsection{Relaciones binarias}
\textbf{Dirige}\\
Es una relación binaria entre Trabajo y Empleado. Hay una participación de 0 a N por parte de Empleado y de 1 a N por parte de trabajo.\\\\
\textbf{Requiere}\\
Es una relacion binaria entre Trabajo y Bien, tiene un atributo descriptivo llamado cantidad y las participaciones son de 0 a N por parte de las dos entidades.\\\\
\textbf{Asociado}\\
Es una relación binaria entre Carrocería y Compra. La participación es de 1 a N por parte de Carrocería y de 0 a N por parte de Compra.\\\\

\subsection{Relaciones ternarias y agregación}
No contamos con relaciones ternarias ni de agragación.


\section{Diccionario de datos}
\subsection{ContactoCliente}
\begin{table}[H]
  \footnotesize
  \begin{tabular}{|c|c|l|l|c|}
  \hline
  \textbf{Campo} & \textbf{Tipo de dato} & \multicolumn{1}{c|}{\textbf{PK}} & \multicolumn{1}{c|}{\textbf{FK}} & \textbf{Descripción}                                 \\ \hline
  dni            & int                   & \multicolumn{1}{c|}{X}           & \multicolumn{1}{c|}{}            & DNI de la persona                                    \\ \hline
  nombre         & varchar               & \multicolumn{1}{c|}{}            &                                  & Nombre de la persona de contacto                     \\ \hline
  celular        & int                   &                                  &                                  & Celular de la persona de contacto                    \\ \hline
  correo         & varchar               &                                  &                                  & Correo electrónico de la persona de contacto         \\ \hline
  Cliente.RUC    & bigint                &                                  & \multicolumn{1}{c|}{X}           & RUC de la empresa vinculada a la persona de contacto \\ \hline
  \end{tabular}
\end{table}

\subsection{ContactoProveedor}
\begin{table}[H]
  \footnotesize
  \begin{tabular}{|c|c|l|l|c|}
  \hline
  \textbf{Campo} & \textbf{Tipo de dato} & \multicolumn{1}{c|}{\textbf{PK}} & \multicolumn{1}{c|}{\textbf{FK}} & \textbf{Descripción}                                 \\ \hline
  dni            & int                   & \multicolumn{1}{c|}{X}           & \multicolumn{1}{c|}{}            & DNI de la persona                                    \\ \hline
  nombre         & varchar               & \multicolumn{1}{c|}{}            &                                  & Nombre de la persona de contacto                     \\ \hline
  celular        & int                   &                                  &                                  & Celular de la persona de contacto                    \\ \hline
  correo         & varchar               &                                  &                                  & Correo electrónico de la persona de contacto         \\ \hline
  Proveedor.RUC  & bigint                &                                  & \multicolumn{1}{c|}{X}           & RUC de la empresa vinculada a la persona de contacto \\ \hline
  \end{tabular}
\end{table}

\subsection{Cliente}
\begin{table}[H]
  \footnotesize
  \begin{tabular}{|c|c|c|c|c|}
  \hline
  \textbf{Campo} & \textbf{Tipo de dato} & \textbf{PK} & \textbf{FK} & \textbf{Descripción}                   \\ \hline
  RUC            & bigint                & X           &             & RUC de la empresa vinculada al cliente \\ \hline
  ubicacion      & varchar               &             &             & RUC de la empresa vinculada al cliente \\ \hline
  razonSocial    & varchar               &             &             & RUC de la empresa vinculada al cliente \\ \hline
  \end{tabular}
\end{table}

\subsection{Proveedor}
\begin{table}[H]
  \footnotesize
  \begin{tabular}{|c|c|l|l|c|}
  \hline
  \textbf{Campo} & \textbf{Tipo de dato} & \multicolumn{1}{c|}{\textbf{PK}} & \multicolumn{1}{c|}{\textbf{FK}} & \textbf{Descripción}                                \\ \hline
  RUC            & bigint                & \multicolumn{1}{c|}{X}           & \multicolumn{1}{c|}{}            & RUC de la empresa vinculada al cliente              \\ \hline
  ubicacion      & varchar               &                                  &                                  & RUC de la empresa vinculada al cliente              \\ \hline
  razonSocial    & varchar               &                                  &                                  & RUC de la empresa vinculada al cliente              \\ \hline
  descripcion    & varchar               &                                  &                                  & Descripción del proveedor                           \\ \hline
  esDeProducto   & bool                  &                                  &                                  & Indica si un proveedor brinda servicios o productos \\ \hline
  \end{tabular}
\end{table}

\subsection{Empleado}
\begin{table}[H]
  \footnotesize
  \begin{tabular}{|c|c|l|l|c|}
  \hline
  \textbf{Campo} & \textbf{Tipo de dato} & \multicolumn{1}{c|}{\textbf{PK}} & \multicolumn{1}{c|}{\textbf{FK}} & \textbf{Descripción}                   \\ \hline
  dni            & int                   & \multicolumn{1}{c|}{X}           &                                  & DNI del empleado                       \\ \hline
  nombre         & varchar               &                                  &                                  & Nombre del empleado                    \\ \hline
  celular        & int                   &                                  &                                  & Celular del empleado                   \\ \hline
  correo         & varchar               &                                  &                                  & Correo electrónico del empleado        \\ \hline
  direccion      & varchar               &                                  &                                  & Dirección del empleado                 \\ \hline
  nacimiento     & date              &                                  &                                  & Fecha de nacimiento del empleado       \\ \hline
  nCuenta        & bigint                &                                  &                                  & Número de cuenta bancaria del empleado \\ \hline
  area           & varchar               &                                  &                                  & Área en donde trabaja el empleado      \\ \hline
  cargo          & varchar               &                                  &                                  & Cargo que ocupa el empleado            \\ \hline
  sueldo         & double precision      &                                  &                                  & Sueldo que recibe el empleado          \\ \hline
  \end{tabular}
\end{table}

\subsection{Dirige}
\begin{table}[H]
  \footnotesize
  \begin{tabular}{|c|c|c|c|c|}
  \hline
  \textbf{Campo} & \textbf{Tipo de dato} & \textbf{PK} & \textbf{FK} & \textbf{Descripción}         \\ \hline
  Empleado.dni   & int                   & X           & X           & DNI del empleado asociado    \\ \hline
  Trabajo.fecha  & datetime              & X           & X           & Fecha del trabajo asociado   \\ \hline
  Trabajo.numero & tinyint               & X           & X           & Número de trabajos asociados \\ \hline
  \end{tabular}
\end{table}

\subsection{Trabajo}
\begin{table}[H]
  \footnotesize
  \begin{tabular}{|c|c|l|l|c|}
  \hline
  \textbf{Campo} & \textbf{Tipo de dato} & \multicolumn{1}{c|}{\textbf{PK}} & \multicolumn{1}{c|}{\textbf{FK}} & \textbf{Descripción}                      \\ \hline
  fecha          & datetime              & \multicolumn{1}{c|}{X}           &                                  & Fecha de ejecución del trabajo            \\ \hline
  numero         & tinyint               & \multicolumn{1}{c|}{X}           &                                  & Numero de trabajo por día (es secuencial) \\ \hline
  descripcion    & varchar               &                                  &                                  & Descripción del trabajo                   \\ \hline
  horas          & real                  &                                  &                                  & Horas de duración del trabajo             \\ \hline
  Maquina.codigo & serial                &                                  & \multicolumn{1}{c|}{X}           & Código de la máquina asociada al trabajo  \\ \hline
  \end{tabular}
\end{table}

\subsection{Máquina}
\begin{table}[H]
  \footnotesize
  \begin{tabular}{|c|c|c|l|c|}
  \hline
  \textbf{Campo} & \textbf{Tipo de dato} & \textbf{PK}           & \multicolumn{1}{c|}{\textbf{FK}} & \textbf{Descripción}      \\ \hline
  codigo         & int                   & X                     &                                  & Código de la máquina      \\ \hline
  descripcion    & varchar               & \multicolumn{1}{l|}{} &                                  & Descripción de la máquina \\ \hline
  \end{tabular}
\end{table}

\subsection{Compra}
  \begin{table}[H]
  \footnotesize
  \begin{tabular}{|c|c|c|c|c|}
  \hline
  \textbf{Campo}   & \textbf{Tipo de dato} & \textbf{PK}           & \textbf{FK}           & \textbf{Descripción}                              \\ \hline
  nFactura         & int                   & X                     &                       & Número de factura asociada a la compra            \\ \hline
  Bien.descripcion & varchar               & X                     & X                     & Descripción del bien asociado a la compra         \\ \hline
  fecha            & date              &                       & \multicolumn{1}{l|}{} & Fecha de la compra                                \\ \hline
  precioUnitario   & double precision      & \multicolumn{1}{l|}{} & \multicolumn{1}{l|}{} & Precio unitario de la compra                      \\ \hline
  cantidad         & smallint              &                       & \multicolumn{1}{l|}{} & Cantidad de la compra                             \\ \hline
  Proveedor.RUC    & bigint                & \multicolumn{1}{l|}{} & X                     & RUC de la empresa proveedora asociada a la compra \\ \hline
  \end{tabular}
\end{table}

\subsection{Requiere}
  \begin{table}[H]
  \footnotesize
  \begin{tabular}{|c|c|c|c|c|}
  \hline
  \textbf{Campo}   & \textbf{Tipo de dato} & \textbf{PK}           & \textbf{FK}           & \textbf{Descripción}                      \\ \hline
  Bien.descripcion & varchar               & X                     & X                     & Descripción del bien asociado             \\ \hline
  Trabajo.fecha    & datetime              & X                     & X                     & Fecha de ejecución del trabajo            \\ \hline
  Trabajo.numero   & tinyint               & X                     & X                     & Numero de trabajo por día (es secuencial) \\ \hline
  cantidad         & smallint              & \multicolumn{1}{l|}{} & \multicolumn{1}{l|}{} & Cantidad utilizada del bien en el trabajo \\ \hline
  \end{tabular}
\end{table}

\subsection{Carrocería}
\begin{table}[H]
  \footnotesize
  \begin{tabular}{|c|c|l|l|c|}
  \hline
  \textbf{Campo} & \textbf{Tipo de dato} & \multicolumn{1}{c|}{\textbf{PK}} & \multicolumn{1}{c|}{\textbf{FK}} & \textbf{Descripción}                           \\ \hline
  id             & serial                & \multicolumn{1}{c|}{X}           & \multicolumn{1}{c|}{}            & Numero serial que identifica a la carrocería   \\ \hline
  placa          & varchar               & \multicolumn{1}{c|}{}            & \multicolumn{1}{c|}{}            & Placa asociada a la carrocería                 \\ \hline
  vin            & varchar               & \multicolumn{1}{c|}{}            & \multicolumn{1}{c|}{}            & Vin asociado a la carrocería                   \\ \hline
  categoria      & varchar               & \multicolumn{1}{c|}{}            &                                  & Categoría a la que pertenece la carrocería     \\ \hline
  carroceria     & varchar               & \multicolumn{1}{c|}{}            &                                  & Tipo de carrocería                             \\ \hline
  largo          & double precision      & \multicolumn{1}{c|}{}            &                                  & Largo de la carrocería                         \\ \hline
  ancho          & double precision      &                                  &                                  & Ancho de la carrocería                         \\ \hline
  alto           & double precision      &                                  &                                  & Alto de la carrocería                          \\ \hline
  nEjes          & smallint              &                                  &                                  & Número de ejes de la carrocería                \\ \hline
  nLlantas       & smallint              &                                  &                                  & Número de llantas de la carrocería             \\ \hline
  color          & varchar               &                                  &                                  & Color de la carrocería                         \\ \hline
  cargaUtil      & double precision      &                                  &                                  & Carga útil de la carrocería                    \\ \hline
  pesoNeto       & double precision      &                                  &                                  & Peso neto de la carrocería                     \\ \hline
  Cliente.RUC    & bigint                &                                  & \multicolumn{1}{c|}{X}           & RUC de la empresa asociada a la carrocería     \\ \hline
  nFactura       & int                   &                                  &                                  & Número de factura de la carrocería             \\ \hline
  precio         & double precision      &                                  &                                  & Precio de la carrocería                        \\ \hline
  medioPago      & varchar               &                                  &                                  & Medio de pago utilizado                        \\ \hline
  alContado      & bool                  &                                  &                                  & Indica si el pago fue al contado o no          \\ \hline
  fecha          & date              &                                  &                                  & Fecha de inicio de producción de la carrocería \\ \hline
  \end{tabular}
\end{table}

\subsection{Asociado}
\begin{table}[H]
  \footnotesize
  \begin{tabular}{|c|c|c|c|c|}
  \hline
  \textbf{Campo}                       & \textbf{Tipo de dato} & \textbf{PK} & \textbf{FK} & \textbf{Descripción}                       \\ \hline
  Carroceria.id & serial                & X           & X           & Placa asociada a la carrocería             \\ \hline
  Compra.nFactura                      & int                   & X           & X           & Número de factura asociado a la carrocería \\ \hline
  Compra.Bien.descripcion              & varchar               & X           & X           & Descripción del bien asociado a la compra  \\ \hline
  \end{tabular}
\end{table}

\subsection{Bien}
\begin{table}[H]
  \footnotesize
  \begin{tabular}{|c|c|c|l|c|}
  \hline
  \textbf{Campo} & \textbf{Tipo de dato} & \textbf{PK}           & \multicolumn{1}{c|}{\textbf{FK}} & \textbf{Descripción}                              \\ \hline
  descripcion    & varchar               & X                     & \multicolumn{1}{c|}{}            & Descripción del bien                              \\ \hline
  esProducto     & bool                  &                       & \multicolumn{1}{c|}{}            & Indica si es un producto o un servicio            \\ \hline
  unidades       & varchar               &                       &                                  & Unidades en las que se mide el bien (kg, m, etc.) \\ \hline
  stock          & smallint              &                       &                                  & Cantidad de bienes en stock                       \\ \hline
  clasificacion  & varchar               & \multicolumn{1}{l|}{} &                                  & Clasificación                                     \\ \hline
  \end{tabular}
\end{table}


%%%%%%%%%%%%%%%%%%%%%%%%%%%%%%%%%%%%%%%%%%%%%%%%%%%%%%%%%%%%%%%%%%%%%%%%%%%%%%%%%%%%%%%%%%%%%%%%%%%%%%%%%%%%%%%%%%%%%%%%%%%%%%%%%%%%%%%%%%%%%%%%%%%%%%%%%%%%%%%%%%%%%%%%%%%%%%%%%%%%%%%%%%%%%%%%%%%%%%%%%
\chapter{Implementación de la base de datos}

\section{Creación de Tablas en PostgreSQL}
\lstinputlisting[style=NiceStyle, language=sql]{baseDeDatos/create_db.sql}
\section{Carga de datos}
Los datos fueron cargados mediante archivos tsv que fueron generados por nosotros mediante scripts hechos en python. 

\section{Simulación de Datos Faltantes}
Los datos fueron simulados en python, los archivos pueden ser vistos en el \href{https://github.com/mayra-diaz/Remolques-Tramontana-DB}{repositorio en GitHub}.

%%%%%%%%%%%%%%%%%%%%%%%%%%%%%%%%%%%%%%%%%%%%%%%%%%%%%%%%%%%%%%%%%%%%%%%%%%%%%%%%%%%%%%%%%%%%%%%%%%%%%%%%%%%%%%%%%%%%%%%%%%%%%%%%%%%%%%%%%%%%%%%%%%%%%%%%%%%%%%%%%%%%%%%%%%%%%%%%%%%%%%%%%%%%%%%%%%%%%%%%%
\chapter{Optimización y Experimentación}
Se realizarán tres consultas críticas para probar los tiempos de ejecución y la respuesta de la base de datos implementada. Además, estas consultas se ejecutarán en diferentes conjuntos de datos de distintos tamaños, los cuales también serán analizados con diferentes indexaciones (sin índices y con índices) con el propósito de comparar los resultados y lograr obtener resultados sólidos.


\section{Consultas SQL para el experimento}
\subsection{Descripción del tipo de consultas seleccionadas}
\textbf{Consulta 1}: ¿Cuántas horas en total trabajaron los empleados con los X sueldos más altos del área de fabricación?
\textbf{Justificación}: La empresa desea saber las horas remuneradas son equivalentes a las trabajadas.

\textbf{Consulta 2}: ¿Cuáles son los bienes de determinada categoría, que cuentan con un stock mayor a 0 y han sido comprados en el último año por un precio unitario mayor a \textit{y} soles?
\textbf{Justificación}: La empresa desea saber cuáles son los bienes a partir de cierto precio que fueron comprados y no se llegaron a utilizar en su totalidad.

\textbf{Consulta 3}: ¿Qué máquinas han utilizado bienes medidos en unidades \textit{x} por una cantidad total mayor a \textit{y} en una determinada fecha?


\subsection{Implementación de consultas en SQL}
\subsubsection*{Consulta 1}
\lstinputlisting[style=NiceStyle, language=sql]{consultas/Consulta1/c1.sql}

\subsubsection*{Consulta 2}
\lstinputlisting[style=NiceStyle, language=sql]{consultas/Consulta2/c2.sql}

\subsubsection*{Consulta 3}
\lstinputlisting[style=NiceStyle, language=sql]{consultas/Consulta3/c3.sql}



\section{Metodología del experimento}
\begin{enumerate}
    \item Crear una base de datos con cuatro esquemas a partir del mismo modelo relacional que se puede observar en el  \hyperlink{A:MR}{ \textbf{Anexo B}}, que tengan 1K, 10K, 100K y 1M de tuplas, respectivamente. 
    \item Utilizar los archivos \textit{.tsv} generados por el script en \texttt{Python} para la generación de datos y con ellos poblar los esquemas.
    \item Crear índices para cada consulta para poder hacer un análisis posterior de las consultas con y sin índices.
    \item Desactivar la utilización automática de índices por parte de PosgreSQL. Asimismo, se desactivan los índices creados. En la Consulta 1 se dejan los índices default activados para hacer una comparación posterior del tiempo de ejecución. Para lograrlo se ejecutan los siguientes comandos:
    \lstinputlisting[style=NiceStyle, language=sql]{consultas/indexes.sql}
    \item Se procede a ejecutar las consultas sin índices con la sentencia \texttt{EXPLAIN ANALYZE} para obtener información acerca de cada consulta y la manera en la que PostgreSQL lo ejecutó. La información obtenida es guardada.
    \item Se limpia la caché entre cada consulta realizada.
    \item Una vez se ha completado la obtención de datos sin índices, se procede a realizar el mismo procedimiento pero con los índices prendidos. Para lograrlo se ejecutan los siguientes comandos:
    \lstinputlisting[style=NiceStyle, language=sql]{consultas/indexesON.sql}
    \item Se realiza el proceso de ejecución de consultas con \texttt{EXPLAIN ANALYZE} y limpieza de caché de nuevo, esta vez utilizando índices. Guardar los datos obtenidos.
    
    
\end{enumerate}
%%%%%%%%%%%%%%%%%%%%%%%%%%%%%%%%%%%%%%%%%%%%%%%%%%%%%%%%%%%%%%%%%%%%%%%%%%%%%%%%%%%%%%%%%%%%%%%%%%%%%%%%%%%%%%%%%%%%%%%%%%%%%%%%%%%%%%%%%%%%%%%%%%%%%%%%%%%%%%%%%%%%%%%%%%%%%%%%%%%%%%%%%%%%%%%%%%%%%%%%%
\section{Optimización de consultas}
Se crean índices con el fin de reducir los tiempos de las tres consultas en cuestión.
\subsection{Consulta 1}
\lstinputlisting[style=NiceStyle, language=sql]{consultas/Consulta1/createIndex.sql}

\subsection{Consulta 2}
\lstinputlisting[style=NiceStyle, language=sql]{consultas/Consulta2/createIndex.sql}

\subsection{Consulta 3}
\lstinputlisting[style=NiceStyle, language=sql]{consultas/Consulta3/createIndex.sql}

\newpage
\subsection{Planes de índices para Consulta 1}
Se muestran los diagramas internos de ejecución generados por PgAdmin. Posteriormente se muestran los resultados obtenidos al ejecutar la consulta con \texttt{EXPLAIN ANALYZE}.
\subsubsection{Ejecución con índices por default (PK)}
\begin{figure}[!h]
  \centering
  \includegraphics[width=300px]{consultas/Consulta1/SI.png}
  \caption{Ejecución de la consulta 1 con índices por default}
\end{figure}
\subsubsection{Ejecución con índices creados}
\begin{figure}[!h]
  \centering
  \includegraphics[width=300px]{consultas/Consulta1/CI.png}
  \caption{Ejecución de la consulta 1 con índices creados}
\end{figure} 

\subsubsection{Explicación y análisis - Con índices por default y con índices creados}
\begin{enumerate}
    \item Ejecución para 1000 datos
    \lstinputlisting[style=TableStyle]{consultas/Consulta1/1000.txt}
    \item Ejecución para 10000 datos
    \lstinputlisting[style=TableStyle]{consultas/Consulta1/10000.txt}
    \item Ejecución para 100000 datos
    \lstinputlisting[style=TableStyle]{consultas/Consulta1/100000.txt}
    \item Ejecución para 1000000 datos
    \lstinputlisting[style=TableStyle]{consultas/Consulta1/1000000.txt}
\end{enumerate}

\subsection{Planes de índices para Consulta 2}
Se muestran los diagramas internos de ejecución generados por PgAdmin.Posteriormente se muestran los resultados obtenidos al ejecutar la consulta con EXPLAIN ANALYZE.
\subsubsection{Ejecución sin índices}
\begin{figure}[!h]
  \centering
  \includegraphics[width=300px]{consultas/Consulta2/SI.png}
  \caption{Ejecución de la consulta 2 sin índices}
\end{figure}
\subsubsection{Ejecución con índices creados}
\begin{figure}[!h]
  \centering
  \includegraphics[width=300px]{consultas/Consulta2/CI.png}
  \caption{Ejecución de la consulta 2 con índices creados}
\end{figure} 

\subsubsection{Explicación y análisis - Sin índices y con índices creados}
\begin{enumerate}
    \item Ejecución para 1000 datos
    \lstinputlisting[style=TableStyle]{consultas/Consulta2/1000.txt}
    \item Ejecución para 10000 datos
    \lstinputlisting[style=TableStyle]{consultas/Consulta2/10000.txt}
    \item Ejecución para 100000 datos
    \lstinputlisting[style=TableStyle]{consultas/Consulta2/100000.txt}
    \item Ejecución para 1000000 datos
    \lstinputlisting[style=TableStyle]{consultas/Consulta2/1000000.txt}
\end{enumerate}

\subsection{Planes de índices para Consulta 3}
Se muestran los diagramas internos de ejecuci ́on generados por PgAdmin.Posteriormente se muestran los resultados obtenidos al ejecutar la consulta con EXPLAIN ANALYZE.
\subsubsection{Ejecución sin índices}
\begin{figure}[!h]
  \centering
  \includegraphics[width=300px]{consultas/Consulta3/SI.png}
  \caption{Ejecución de la consulta 3 sin índices}
\end{figure}
\newpage
\subsubsection{Ejecución con índices creados}
\begin{figure}[!h]
  \centering
  \includegraphics[width=300px]{consultas/Consulta3/CI.png}
  \caption{Ejecución de la consulta 3 con índices creados}
\end{figure} 

\subsubsection{Explicación y análisis - Sin índices y con índices creados}
\begin{enumerate}
    \item Ejecución para 1000 datos
    \lstinputlisting[style=TableStyle]{consultas/Consulta3/1000.txt}
    \item Ejecución para 10000 datos
    \lstinputlisting[style=TableStyle]{consultas/Consulta3/10000.txt}
    \item Ejecución para 100000 datos
    \lstinputlisting[style=TableStyle]{consultas/Consulta3/100000.txt}
    \item Ejecución para 1000000 datos
    \lstinputlisting[style=TableStyle]{consultas/Consulta3/1000000.txt}
\end{enumerate}





%%%%%%%%%%%%%%%%%%%%%%%%%%%%%%%%%%%%%%%%%%%%%%%%%%%%%%%%%%%%%%%%%%%%%%%%%%%%%%%%%%%%%%%%%%%%%%%%%%%%%%%%%%%%%%%%%%%%%%%%%%%%%%%%%%%%%%%%%%%%%%%%%%%%%%%%%%%%%%%%%%%%%%%%%%%%%%%%%%%%%%%%%%%%%%%%%%%%%%%%%

\section{Plataforma de Pruebas}
Los datos registrados en la siguiente tabla nos permitirán, en próximas investigaciones, comparar nuestros resultados. Estos valores nos servirán como métricas referenciales para alinear el desarrollo de proyectos futuros.


\begin{table}[h]
\footnotesize
\begin{tabular}{|c|c|}
\hline
\textbf{Item}                                   & \textbf{Detalle}              \\ \hline
Sistema Operativo                               & Windows 10 Pro                \\ \hline
RAM                                             & 8 Gb                          \\ \hline
CPU                                             & Core i5-3470                  \\ \hline
HDD Protocolo de Transferencia                  & SATA                          \\ \hline
Modelo de disco sólido                          & ST1000DM003-1CH162 ATA Device \\ \hline
HDD Capacidad                                   & 193 Gb                        \\ \hline
HDD Tasa de Transferencia (Lectura / Escritura) & 540 Mbps / 410 Mbps           \\ \hline
PostgreSQL                                      & 12.4                          \\ \hline
PgAdmin                                         & Versión 4                     \\ \hline
\end{tabular}
\caption{Tabla de plataforma de pruebas para consulta 1}
\end{table}

\begin{table}[]
\footnotesize
\begin{tabular}{|c|c|}
\hline
\textbf{Item}                                   & \textbf{Detalle}            \\ \hline
Sistema Operativo                               & Windows 10                  \\ \hline
RAM                                             & 12 Gb                       \\ \hline
CPU                                             & Core i5-10210U              \\ \hline
HDD Protocolo de Transferencia                  & SATA                        \\ \hline
Modelo de disco sólido                          & KINGSTON RBUSNS8154P3128GJ1 \\ \hline
HDD Capacidad                                   & 1 Tb                        \\ \hline
HDD Tasa de Transferencia (Lectura / Escritura) & 540 Mbps / 410 Mbps         \\ \hline
PostgreSQL                                      & 12.4                        \\ \hline
PgAdmin                                         & Versión 4                   \\ \hline
\end{tabular}
\caption{Tabla de plataforma de pruebas para consulta 2}
\end{table}

\begin{table}[]
\footnotesize
\begin{tabular}{|c|c|}
\hline
\textbf{Item}                                   & \textbf{Detalle}            \\ \hline
Sistema Operativo                               & Ubuntu 20.01.1                  \\ \hline
RAM                                             & 8 Gb                       \\ \hline
CPU                                             & Core™ i7-8550U              \\ \hline
SSD Protocolo de Transferencia                  & SATA                        \\ \hline
Modelo de disco sólido                          & Micron_1100_MTFD \\ \hline
SSD Capacidad                                   & 256 GB                        \\ \hline
SSD Tasa de Transferencia (Lectura / Escritura) & 500 Mbps / 530 Mbps         \\ \hline
PostgreSQL                                      & 12.4                        \\ \hline
PgAdmin                                         & Versión 4                   \\ \hline
\end{tabular}
\caption{Tabla de plataforma de pruebas para consulta 3}
\end{table}



\newpage
\section{Medición de tiempos}
Se realizaron las consultas por separado, repitiendo la ejecución un total de quince veces por cada esquema (1K, 10K, 100K y 1M).
Los resultados obtenidos fueron resumidos y analizados en tablas y gráficos.\\

Para tener tiempos más exactos realizamos 15 veces cada consulta. Los resultados pueden ser vistos \href{https://docs.google.com/spreadsheets/d/1VTr49dQ2VYkrrlQ-rBszSXFpJIpz7h5ejWiqU3tfeNU/edit?usp=sharing}{\textbf{aquí}}.


\subsection{Sin índices}
Se ejecutaron las consultas sin hacer uso de índices, y se halló el promedio de cada una.
\begin{table}[h]
\begin{tabular}{|c|c|c|c|c|}
\hline
\multicolumn{5}{|c|}{\textbf{SIN ÍNDICES: Tiempo de ejecución (ms) vs tamaño del dataset}} \\ \hline
\multicolumn{1}{|l|}{}   & \textbf{1K}   & \textbf{10K}   & \textbf{100K}   & \textbf{1M}  \\
\textbf{Consulta 1*}    & 1.165         & 9.445          & 122.262         & 1745.922     \\
\textbf{Consulta 2}      & 0.603         & 7.023          & 124.800         & 472.665      \\
\textbf{Consulta 3}      & 19.508        & 126.688        & 712.887         & 3267.110     \\ \hline
\end{tabular}
\caption{Tabla de consultas sin índice. *La consulta 1 fue realizada con índices default}
\label{tab:my-table}
\end{table}
\newpage
\subsection{Con índices}
Después de finalizar la ejecución de consultas sin índices, procedemos a activar los índices, recopilando los datos obtenidos y hallando un promedio.
\begin{table}[h]
\begin{tabular}{|c|c|c|c|c|}
\hline
\multicolumn{5}{|c|}{\textbf{CON ÍNDICES: Tiempo de ejecución (ms) vs tamaño del dataset}} \\ \hline
\multicolumn{1}{|l|}{}   & \textbf{1K}   & \textbf{10K}   & \textbf{100K}   & \textbf{1M}  \\
\textbf{Consulta 1}      & 0.1321        & 0.177          & 0.426           & 3.931        \\
\textbf{Consulta 2}      & 0.21          & 1.490          & 11.454          & 323.179      \\
\textbf{Consulta 3}      & 3.5797        & 39.842         & 285.893         & 1550.183     \\ \hline
\end{tabular}
\caption{Tabla de consultas con índice}
\label{tab:my-table}
\end{table}

\subsection{Mediciones independientes}
Para poder realizar un análisis más profundo de cada consulta respecto a los distintos datasets, se muestran los datos obtenidos por cada consulta.

\begin{table}[h]
\begin{tabular}{|c|c|c|c|c|}
\hline
\multicolumn{5}{|c|}{\textbf{CONSULTA 1: Tiempo de ejecución (ms) vs tamaño del dataset}} \\ \hline
                          & \textbf{1K}   & \textbf{10K}  & \textbf{100K}  & \textbf{1M}  \\
\textbf{Índice default}   & 1.165         & 9.445         & 122.262        & 1745.922    \\
\textbf{Con Índice}       & 0.132         & 0.177         & 0.426          & 3.931        \\ \hline
\end{tabular}
\caption{Tabla de ejecución de consulta 1 con índice creado y con índice default}
\label{tab:my-table}
\end{table}

\begin{table}[h]
\begin{tabular}{|c|c|c|c|c|}
\hline
\multicolumn{5}{|c|}{\textbf{CONSULTA 2: Tiempo de ejecución (ms) vs tamaño del dataset}} \\ \hline
                       & \textbf{1K}   & \textbf{10K}   & \textbf{100K}   & \textbf{1M}   \\
\textbf{Sin índice}    & 0.603         & 7.023          & 124.800         & 472.665       \\
\textbf{Con Índice}    & 0.214         & 1.490          & 11.454          & 323.179       \\ \hline
\end{tabular}
\caption{Tabla de ejecución de consulta 2 con índice y sin índice}
\label{tab:my-table}
\end{table}

\begin{table}[h]
\begin{tabular}{|c|c|c|c|c|}
\hline
\multicolumn{5}{|c|}{\textbf{CONSULTA 3: Tiempo de ejecución (ms) vs tamaño del dataset}} \\ \hline
                       & \textbf{1K}   & \textbf{10K}   & \textbf{100K}   & \textbf{1M}   \\
\textbf{Sin índice}    & 19.508        & 126.688        & 712.887         & 3267.110      \\
\textbf{Con Índice}    & 3.580         & 39.842         & 285.893         & 1550.183     \\ \hline
\end{tabular}
\caption{Tabla de ejecución de consulta 3 con índice y sin índice}
\label{tab:my-table}
\end{table}

\section{Resultados}

\subsection{Consulta 1}

\begin{figure}[!h]
  \centering
  \includegraphics[scale=0.7]{consultas/Consulta1/grafico1.png}
\end{figure} 

Para la consulta 1, en la que se compara el desempeño cuando se utiliza un índice creado y cuando se utilizan los índices default, se puede apreciear una clara diferencia. En general, con índice propio se comporta de una manera extremadamente eficaz, incluso con el dataset de 1M. Por otro lado, el desempeño con los índices default es bueno hasta los 10K, pero en adelante el tiempo de ejecución aumenta notablemente.

\newpage
\subsection{ Consulta 2}

\begin{figure}[h!]
  \centering
  \includegraphics[scale=0.7]{consultas/Consulta2/grafico2.png}
\end{figure} 

Haciendo un análisis de comportamieto del tiempo de ejecución de la consulta 2 con índices y sin índices, podemos apreciar que las diferencias son cada vez mayores a medida que aumenta el tamaño de nuestro dataset. Viendo el comportamiento de los datos por iteración para el dataset de 100k, podemos ver que en este caso ocurre la mayor diferencia entre la consulta con índices y sin indíces. Cabe resaltar que se usaron 4 índices para esta consulta, dos B+ y dos Hash.

\newpage
\subsection{ Consulta 3}

\begin{figure}[!h]
  \centering
  \includegraphics[scale=0.7]{consultas/Consulta3/grafico3.png}
\end{figure} 

Podemos ver que la versión optimizada con índices se empieza a distinguir notablemente de la otra con el crecimiento de la data. Cuando ejecutamos las consultas con pocos datos la diferencia no se percive mucho, sin embargo para que nuestras consultas sean buenas y escalables debemos optimizarla. En este caso vemos cómo el uso de los índices adecuados realmente hacen la diferencia.

\newpage
\section{Análisis y Discusión}

Para poder analizar el desempeño general de las consultas, se generó una tabla que muestra el promedio de tiempo de acuerdo a la indexación, analizado en cada dataset.
\begin{table}[h]
\begin{tabular}{|c|c|c|c|c|}
\hline
\multicolumn{5}{|c|}{\textbf{PROMEDIO: Tiempo de ejecución (ms) vs tamaño del dataset}} \\ \hline
                      & \textbf{1K}   & \textbf{10K}   & \textbf{100K}   & \textbf{1M}  \\
\textbf{Sin Índice}   & 9.473         & 47.719         & 319.983         & 1828.566     \\
\textbf{Con Índice}   & 1.31          & 13.836         & 99.258          & 625.764      \\ \hline
\end{tabular}
\caption{Tabla de promedio de las tres consultas}
\label{tab:my-table}
\end{table}

Observando los datos numéricos, se puede ver que al indexar una tabla se mejora bastante el tiempo de ejecución de una consulta. Sin embargo, sería óptimo analizar esta información de una manera más gráfica, por lo que se hace uso de gráficas de dispersión.

\newpage
\begin{figure}[!h]
  \centering
  \includegraphics[scale=0.7]{consultas/graficoPromedioBase.png}
\end{figure} 

La gráfica anterior permite visualizar que para datasets de pocos datos, no existe una diferencia considerable en el tiempo de ejecución. No obstante, conforme aumenta el tamaño de las tablas, de 100K a más, el desempeño con índices es notablemente más óptimo. Esto comprueba los propósitos de la aplicación de índices, que las consultas frecuentes sean más rápidas.

Es de mucho interés conocer el comportamiento con cantidades de datos más grandes, por lo que se plantea un modelo de tendencias polinómico de grado 2 para proyectar el tiempo de ejecución con distintas cantidades de datos.
%%%%%%%%%%%%%%%%%%%%%%%%%%%%%%%%%%%%%%%%%%%%%%%%%%%%%%%%%%%%%%%%%%%%%%%%%%%%%%%%%%%%%%%%%%%%%%%%%%%%%%%%%%%%%%%%%%%%%%%%%%%%%%%%%%%%%%%%%%%%%%%%%%%%%%%%%%%%%%%%%%%%%%%%%%%%%%%%%%%%%%%%%%%%%%%%%%%%%%%%%
\newpage
\begin{figure}[!h]
  \centering
  \includegraphics[scale=0.7]{consultas/graficoPromedioPolinomio.png}
\end{figure} 

La gráfica muestra claramente que las líneas de tendencia polinómica se ajustan con bastante precisión al crecimiento real de las curvas de datos. Debido a que las ecuaciones que describen el comportamiento de las curvas generada responden
a polinomios de grado 2, esto indica que el crecimiento del tiempo de ejecución de las consultas aumenta conforme se tienen más datos en los datasets de manera. Las ecuaciones de tendencia son las siguientes:

\begin{equation}
    SI = F(x) = 368x^2 - 530x + 59.6
\end{equation}
\begin{equation}
    CI = G(x) = 128x^2 - 190x + 19.7
\end{equation}
Desarrollando un poco más las ecuaciones de tendencia, se llegan a las derivadas de cada una:
\begin{equation}
    SI' = F'(x) = 736x - 530
\end{equation}
\begin{equation}
    CI' = G'(x) = 256x - 190
\end{equation}
Estas nuevas ecuaciones representan la razón de cambio del tiempo de ejecución de una consulta respecto a la variación del tamaño del dataset. Si calculamos las razones de cambio en el límite del experimento (1M), la ecuación (5.3) indica que la razón de cambio para consultas sin índice es de $736\cdot10^6\frac{ms}{registro}$. Por otra parte, la ecuación (5.4) indica que la razón de cambio para consultas con índice es de $256\cdot10^6\frac{ms}{registro}$. De esta manera, se aprecia que las consultas (para el millón de registros) sin índice son 287.5\% más costosas que las que si tienen índices.



\chapter{Conclusiones}
\begin{itemize}
    \item Al momento de realizar el modelo entidad relación es fundamental familiarizarse completamente con el cliente. Establecer las reglas semánticas y entender el flujo de datos permite diseñar mejor y óptimamente la base de datos a implementar. El modelo relacional nos ayuda a poder determinar la veracidad de nuestro modelo y mejorarlo en caso necesite.
    
    \item El presente proyecto hizo uso de PostgresSQL para la implementación de la base de datos y la experimentación de las consultas. Este es el gestor de base de datos sobre el que se realizaron todas las operaciones del proyecto.
    
    \item El análisis numérico de tendencias polinomialesnos permite afirmar con más fundamentos que en efecto las tres consultas indexadas son más eficientes que las no indexadas. De igual manera, es vital saber a qué se le está aplicando índices y si en verdad es necesario, pues toda mejora en complejidad de ejecución tiene un costo adicional en espacio de memoria.
    
    \item En base a la experimentación podemos demostrar que el uso de índices es fundamental en el uso de bases de datos. Sobretodo si queremos que esta sea escalable, pues reduce los tiempos y optimiza recursos.
    
    \item No podemos afirmar que tán óptima es nuestra consulta en base a una base de datos con pocos datos, pues no se distingue bien y tampoco asegura que sea escalable ni real en el mercado.
\end{itemize}


\appendix
\appendixpage
\chapter{Entidad-Relación}
\label{A::MER}
\begin{sidewaysfigure}[ht]
    \centering
    \includegraphics[width=\textwidth]{Entidad-Relacion-RT.pdf}
\end{sidewaysfigure}

\chapter{ Relacional}
\label{A::MR}
\begin{sidewaysfigure}[ht]
    \centering
    \includegraphics[width=\textwidth]{Modelo-Relacional-RT.pdf}
\end{sidewaysfigure}
  
  \chapter{Enlaces}
  \begin{itemize}
      \item Repositorio: \url{https://github.com/mayra-diaz/Remolques-Tramontana-DB}
      \item Carpeta: \url{https://bit.ly/3a5kh2U}
  \end{itemize}
  \pagebreak
\end{document}
